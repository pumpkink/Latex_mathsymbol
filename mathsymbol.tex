\documentclass{article}
  \usepackage{amsmath}
  \usepackage{amssymb}
\begin{document}
  The Newton's second law is F=ma.

  The Newton's second law is $F=ma$.

  The Newton's second law is
  $$F=ma$$

  The Newton's second law is
  \[F=ma\]

  Greek Letters $\eta$ and $\mu$

  Fraction $\frac{a}{ab}$

  Power $a^b$

  Subscript $a_b, \mu_{max}, \mu_{min}$

  Derivate $\frac{\partial y}{\partial t} $

  Vector $\vec{n}$

  Bold $\mathbf{n}$

  To time differential $\dot{F}$
  
  Funktionaler Bereich: $\forall x \in X, \quad\exists y\leq \epsilon$
  
  Greek letters: $\alpha, A, \beta, B, \gamma, \Gamma, \pi, \Pi, \phi, \varphi, \mu, \Phi$
  
  Operator: $$\cos(2\theta) = \cos^2\theta - \sin^2\theta$$
            $$\lim_{x\to\infty} f(x) = 0 $$
  Power and indices: $k_{n+1} = n^2 + k_n^2 - k_{n-1}$\\
                     $$f(n) = n^5 + 4n^2 + 2 |_{n=17}$$
  Fractions and Binomials:  
  $$\frac{n!}{k!(n-k)!} = \binom{n}{k}$$\\
  $$\frac{\frac{1}{x}+\frac{1}{y}}{y-z} = x^\frac{1}{2} $$\\ 
  $$\sqrt[n]{1+x+x^2+x^3+\frac{x}{2}+\ldots}$$
  Sums and integrals:
  $$\displaystyle\sum_{i=1}^{10} t_i$$\\
  $$\int_0^\infty\mathrm{e}^{-x}\,\mathrm{d}x$$
  $$\sum_{\substack{0<i<m\\0<j<n}} P(i,j)$$
  Automatic sizing
  $$\left(\frac{x^2}{y^3}\right)$$
  $$P\left(A=2\middle|\frac{A^2}{B}>4\right)$$
  $$\left.\frac{x^3}{3}\right|_0^1$$   
  Typesetting intervals
  $$x\in[-1,1]$$
  Matrix (lcr here means left, center or right for each column)
  \[
    \left[
      \begin{array}{lcr}
        a1 & b22 & c333 \\
        a2 & b23 & f6
      \end{array}
    \right]
  \]
$$ A_{m,n} = 
  \begin{pmatrix}
  	a_{1,1} & a_{1,2} & \cdots & a_{1,n}\\
  	a_{2,1} & a_{2,2} & \cdots & a_{2,n}\\
  	\vdots  & \vdots  & \vdots & \vdots \\
  	a_{m,1} & a_{m,2} & \cdots & a_{m,n}
  \end{pmatrix}$$
  
$$ M = \bordermatrix{~ & x & y \cr
	                 A & 1 & 0 \cr
	                 B & 0 & 1 \cr}$$ 
A matrix in text must be set smaller:
	                $\bigl(\begin{smallmatrix}
	                a&b \\ c&d
	                \end{smallmatrix} \bigr)$ to not increase leading in a portion of text.\\
Equations(here \& is the symbol for aligning different rows)
\begin{align}
  a+b&=c\\
  d&=e+f+g
\end{align}
Controlling horizontal spacing: 
\[ f(n) =
\begin{cases}
n/2       & \quad \text{if } n \text{ is even}\\
-(n+1)/2  & \quad \text{if } n \text{ is odd}\\
\end{cases}
\]\\

\[
  \left\{
    \begin{aligned}
      &a+b=c\\
      &d=e+f+g
    \end{aligned}
  \right.
\]

\end{document}